%%%%%%%%%%%%%%%%%%%%%%%%%%%%%%%%%%%%%%%%%
% University/School Laboratory Report
% LaTeX Template
% Version 3.1 (25/3/14)
%
% This template has been downloaded from:
% http://www.LaTeXTemplates.com
%
% Original author:
% Linux and Unix Users Group at Virginia Tech Wiki 
% (https://vtluug.org/wiki/Example_LaTeX_chem_lab_report)
%
% License:
% CC BY-NC-SA 3.0 (http://creativecommons.org/licenses/by-nc-sa/3.0/)
%
%%%%%%%%%%%%%%%%%%%%%%%%%%%%%%%%%%%%%%%%%

%----------------------------------------------------------------------------------------
%	PACKAGES AND DOCUMENT CONFIGURATIONS
%----------------------------------------------------------------------------------------

\documentclass{article}

\usepackage[version=3]{mhchem} % Package for chemical equation typesetting
\usepackage{siunitx} % Provides the \SI{}{} and \si{} command for typesetting SI units
\usepackage{graphicx} % Required for the inclusion of images
\usepackage{natbib} % Required to change bibliography style to APA
\usepackage{amsmath} % Required for some math elements 
\usepackage[boxed]{algorithm2e} % required for algorithms
\usepackage{fancyhdr}

%\setlength\parindent{0pt} % Removes all indentation from paragraphs; ignore this

\renewcommand{\labelenumi}{\alph{enumi}.} % Make numbering in the enumerate environment by letter rather than number (e.g. section 6)

\pagestyle{fancy}
\fancyhf{}
\lhead{Whitacre School of Engineering}
\rhead{Texas Tech University}
\rfoot{\thepage}
\lfoot{TR 2015-001}

%\usepackage{times} % Uncomment to use the Times New Roman font

%----------------------------------------------------------------------------------------
%	DOCUMENT INFORMATION
%----------------------------------------------------------------------------------------

\title{GoblinCore64: \\ Architectural Specification \\ Technical Report 2015-001} % Title

\author{John D. \textsc{Leidel}} % Author name

\date{\today} % Date for the report

\begin{document}

\maketitle % Insert the title, author and date

\begin{center}
\begin{tabular}{l r}
Version: & 1.0 \\ % Date the experiment was performed
Texas Tech University \\ % Partner names
Data Intensive Scalable Computing Laboratory \\ % Course

\end{tabular}
\end{center}

\newpage

\tableofcontents

\newpage

%----------------------------------------------------------------------------------------
%	SECTION 1
%----------------------------------------------------------------------------------------
\section{Introduction}

\subsection{GC64 Overview}

\subsection{RISC-V ISA Requirements}

The GC64 RISC-V extension functions as both a core RISC-V architectural extension and an extension to the standard 64-bit RISC-V IMAFD specification.  The following minimum architectural extensions are required to utilize GC64.  

\begin{itemize}
\item \textbf{RV64I}:  GC64 requires 64-bit addressing at minimum.  The RV32I memory model is supported as a result.  However, given the use of HMC devices, we require the use of at least 64-bit addressing (and addressing arithmetic).  
\item \textbf{M-Extension}:  GC64 requires the integer multiplication and division for the purpose of efficiently performing address manipulation.  
\item \textbf{A-Extension}:
\end{itemize}

The following architectural extensions are support in the GC64 tasking model.  They are herein named as optional extensions.  

\begin{itemize}
\item \textbf{F-Extension}:  By default, the GC64 task context definition supports the storage of the single precision floating point register values.  If the F-extension is not present, these values are ignored.  
\item \textbf{D-Extension}: By default, the GC64 task context definition supports the storage of the double precision floating point register values.  If the D-extension is not present, these values are ignored.  
\end{itemize}

\subsection{RISC-V SoC Architecture}

%----------------------------------------------------------------------------------------
%	SECTION 2
%----------------------------------------------------------------------------------------
\section{GC64 Instruction Set Extension}

\subsection{GC64 RISC-V Nomenclature}
\subsection{GC64 Register State}
\subsection{Integer Load/Store Instructions}
\subsection{Single-Precision Load/Store Instructions}
\subsection{Double-Precision Load/Store Instructions}
\subsection{Concurrency Instructions}
\subsection{Task Control Instructions}
\subsection{Supervisor Instructions}

%----------------------------------------------------------------------------------------
%	SECTION 3
%----------------------------------------------------------------------------------------
\section{GC64 Task Interface Specification}

\subsection{Overview}

\subsection{Task Context}

\subsection{Task Queuing}

%----------------------------------------------------------------------------------------
%	SECTION 4
%----------------------------------------------------------------------------------------
\section{GC64 Instruction Set Listings}

\framebox[15ex][s]{foobar}

%----------------------------------------------------------------------------------------
%	SECTION 5
%----------------------------------------------------------------------------------------
\section{History and Acknowledgements}

\end{document}